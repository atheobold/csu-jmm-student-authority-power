% Please not use any packages not already included, nor any user defined macros.% That is, expand out and of your macros where they occur using commands that
%  work in this template without modifying the preamble.

\documentclass[11pt]{article}
\usepackage{amsmath,amssymb,eucal}
\usepackage{fancyhdr}
\usepackage{url}
\RequirePackage[pdftex]{hyperref}
\hypersetup{
    colorlinks=true,
    urlcolor=blue,
}

\textwidth 150mm
\hoffset -12mm

\pagestyle{fancy}
\headsep = 15mm
\parindent0pt
\begin{document}
\rhead{{\small The First CSU Mathematical Conference \\ California State University, Northridge \\ November 11--12, 2022, Northridge, California}}



\begin{center}
  {\Large
    {\bf
	Navigating Issues of Power in the Classroom, Facilitating Equitable Group Collaborations
    }
  }
  
  \medskip
  
  {\bf
    Allison S. Theobold
    }
  
  \smallskip

{
    Department of Statistics, \\
    California Polytechnic State University, San Luis Obispo \\
    1 Grand Avenue, San Luis Obispo, CA 93407, USA \\
    \href{mailto:atheobol@calpoly.edu}{atheobol@calpoly.edu} and \url{https://statistics.calpoly.edu/allison-theobold}\\
}
 
\end{center}

\medskip

{\bf Key Words:} Groupwork, Equity, Student Identity

\medskip

For nearly two decades active learning has been elevated by statistics educators as a critical component of the statistics classroom [1]. 
Yet, these conversations have yet to acknowledge how group collaborations fall prey to issues of status—the distribution of power among peers
[2]. Fights over who gets to speak and whose words are recognized are indicative of power and status [3], where students with higher 
status are positioned as credible sources of information, gain and maintain the conversational floor, and have their ideas attended to [4, 5].
Through the lens of group discourse, mathematics education researchers have demonstrated how the gendered and racial microaggressions
students face oppress their opportunity to learn and squelch the development of their mathematical identity [4]---a substantial determinant
of their persistence within the discipline [6]. If we hope for data science to be diverse, \emph{every} student needs to feel they belong within the
discipline. 

In this presentation, I will describe pedagogical tools which actively challenge systems of power in the classroom, creating a more
equitable environment for collaborative group work. I will conclude with a discussion of my experiences using these pedagogical 
tools in an introductory data science course. 
 
\medskip

{\bf References}

\smallskip

[1] GAISE College Report ASA Revision Committee, ``Guidelines for Assessment and Instruction in Statistics Education College Report 2016.''

[2] Cohen, E. G. \& Lotan, R. A. (1995). Producing Equal-Status Interaction in the Heterogeneous Classroom. \emph{American Educational Research Journal}, 32(1), 99–120.

[3] Johnson, B. (2002). \emph{Introducing linguistics}: Vol. 3. \emph{Discourse analysis}. Malden, MA: Blackwell. 

[4] Langer-Osuna, J. M. (2016). The social construction of authority among peers and its implications for collaborative mathematics problem solving. \emph{Mathematical Thinking and Learning}, 18(2), 107-124. 

[5] Engle, R. A., Langer-Osuna, J. M., \& McKinney de Royston, M. (2014). Toward a model of influence in persuasive discussions: Negotiating quality, authority, privilege, and access within a student-led argument. \emph{Journal of the Learning Sciences}, 23(2), 245-268.  

[6] Conference Board of the Mathematical Sciences (2016). \emph{Active learning in post-secondary mathematics education}.

\end{document}
    
    


    
        
  
        
